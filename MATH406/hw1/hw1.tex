\documentclass{article}
\usepackage[utf8]{inputenc}
\usepackage[english]{babel}
\usepackage{amsthm}
\usepackage{amssymb}
\usepackage{amsmath, amsfonts}
\usepackage{extramarks}


\title{HW 1}
\author{Bailey Wickham}
\date\today

\newcounter{homeworkProblemCounter}
\setcounter{homeworkProblemCounter}{1}
\nobreak\extramarks{Problem \arabic{homeworkProblemCounter}}{}\nobreak{}



\newenvironment{homeworkProblem}[1]{
    \section*{Problem #1}
}


\begin{document}
\maketitle

\begin{homeworkProblem}{1}
    Let $V$ be a Real inner product space. Show that if $u,v\in V$ have the same norm, then $u+v$, $u-v$ are orthogonal.
\end{homeworkProblem}

\begin{proof}
    First, we are in a Real product space, so our terms commute. We also know that $\langle u,u \rangle = \langle v,v \rangle $. We will show inner product of $u+v, u-v$ is $0$ to show they are orthogonal
    \begin{gather*}
        \langle u + v, u - v \rangle\\
        \langle u, u-v \rangle + \langle v, u-v \rangle \\
        \langle u,u \rangle - \langle v,v \rangle + \langle u,v \rangle - \langle u,v \rangle \\
        \text{here we know $\langle u,u \rangle = \langle v,v \rangle $ because $u,v$ have the same norm} \\
        \therefore 0 \\
        \therefore \text{$u+v, u-v$ are orthogonal}
    \end{gather*}
\end{proof}

\begin{homeworkProblem}{2}
    Give examples of $u,v$ and an inner product space $V$ where
    \begin{gather*}
        \| u+v \| = \|u\|^2 + \| v \|^2
    \end{gather*}
    despite the fact $u$ is not orthogonal to $v$.
\end{homeworkProblem}

\begin{proof}
    Take our inner product space to be $\mathbb{C}$ under the normal inner product, $u=1, v=1+i$. Our vectors are not orthogonal, $\langle u,v \rangle = 1-i$.
    \begin{align*}
        \| 2 + i\|^2 &= \| 1 \|^2 + \| 1 + i \|^2 \\
        \sqrt{4 - 1}^2 &= \sqrt{1}^2 + \sqrt{2}^2 \\
        3 &= 3
    \end{align*}
\end{proof}

\begin{homeworkProblem}{3}
    Let the vector space $\mathcal{P}_2{(\mathbb{R})}$ be equipped with an inner product given by
    \begin{gather*}
        \langle p,q \rangle = \int_{0}^{1} p(x)q(x)dx
    \end{gather*}
    Find all polynomials that are orthogonal to the polynomial $x$. Is this infinite? Is this a subspace?
\end{homeworkProblem}

\begin{proof}
    To get all orthogonal polynomials, we want to set our inner product equal to $0$.
    \begin{align*}
        0 &= \int_0^1 x(a_0+a_1x+a_2x^2)dx \\
        0 &= \int_0^1 a_0x+a_1x^2+a_2x^3dx \\
    0&=\frac{a_0}{2}x^2 + \frac{a_1}{3}x^3 + \frac{a_2}{4}x^4 \biggr\rvert_0^1 \\
    0&=\frac{a_0}{2} + \frac{a_1}{3} + \frac{a_2}{4}
    \end{align*}
    This is an infinite collection by varying $a_0,a_1,a_2$. This is dim 2 subspace.

\end{proof}

\begin{homeworkProblem}{4}
    Let $V$ be an IPS equipped with $\langle \cdot, \cdot \rangle $, and let $T \in \mathcal{L}(V)$. Show that
    \begin{gather*}
        \langle u, v \rangle _{cousin} = \langle Tu,Tv \rangle
    \end{gather*}
    is an inner product on $V$ iff $T$ is injective
\end{homeworkProblem}

\begin{proof}
    $\implies$ Let $\langle u,v \rangle _{cousin} $ be an inner product.
    We want to show that $0$ is the only vector in the null space. Assume $Tu = 0$. Then $\langle Tu, Tu \rangle =0$ because $\langle \cdot,\cdot \rangle$ is an inner product. This is equivalent to $\langle u,u \rangle_{cousin} = 0$. Since $\langle \cdot, \cdot \rangle_{cousin}$ is an inner product,
    \begin{gather*}
        \therefore u = 0 \\
        \therefore T \text{ is injective}
    \end{gather*}

    $\impliedby$ Let $T$ be injective. We must show $\langle \cdot, \cdot \rangle_{cousin}$ is an inner product. \\
    \textbf{positivity:} Take $\langle v,v \rangle_{cousin} = \langle Tv, Tv  \rangle $. $\langle Tv, Tv \rangle \ge 0$ Since $\langle \cdot, \cdot \rangle$ is an inner product. Therefore $\langle \cdot, \cdot \rangle $ satisfies positivity. \\
    \textbf{definiteness:} Take $\langle v,v \rangle_{cousin} = 0$, therefore $\langle Tv,Tv \rangle = 0$ and $Tv = 0$ since $T$ is injective, $v=0$, and $\langle \cdot,\cdot \rangle_{cousin}$ is definite. \\
    The following properties follow the same structure as the first two and do not require the injectivity of $T$.
    \begin{gather*}
        \therefore \langle \cdot, \cdot \rangle_{cousin} \text{ is an inner product.}
    \end{gather*}

\end{proof}

\begin{homeworkProblem}{Axler 5}
    Suppose $T \in \mathcal{L}(V)$ is such that $\|Tv\| \le \|v\|$ Prove that $T - \sqrt{2}I $ is invertible.
\end{homeworkProblem}

\begin{proof}
    We know $\sqrt{2} $ is an eigenvalue of $T$ iff $null(T-\lambda I) \ne 0$, taking the contrapositive, $null(T - \lambda I) = 0$ iff $\sqrt{2} $ is not an eigenvalue. Assume $\sqrt{2}$ is an eigenvalue. Then
    \begin{gather*}
       \|Tv\| \le \|v\| \\
       \|\sqrt{2}v\| \le \|v\| \\
       \sqrt{2} \|v\| \le \|v\|
    \end{gather*}
    Which is a contradiction. Therefore $\sqrt{2} $ is not an eigenvalue, and $T-\sqrt{2}I $ is invertible.
\end{proof}

\begin{homeworkProblem}{Axler 8}
    Suppose $u, v \in V$ and $\|u\| = \|v\| = 1$ and $\langle u,v  \rangle = 1$. Prove $u=v$
\end{homeworkProblem}
\begin{proof}
    Take $u=v, u-v = 0$ if we take our norm, substituting using our assumptions,
    \begin{align*}
        \|u-v\| &= \langle u,u \rangle - \langle u,v \rangle - \langle v,u \rangle + \langle v,v \rangle \\
        \|u-v\| &= 0
    \end{align*}
\end{proof}

\begin{homeworkProblem}{Axler 11}
    Prove $16 \le (a + b + c + d)(\frac{1}{a} + \frac{1}{b} + \frac{1}{c} + \frac{1}{d})$
\end{homeworkProblem}
\begin{proof}
    Since we are in $\mathbb{R}$, we can take the Cauchy-Schwarz Inequality example in 6.17.a.
    \begin{gather*}
        |x_1y_1+\dots+x_ny_n|^2 \le (x_1^2+\dots+x_n^2)(y_1^2+\dots+y_n^2) \\
        4^2 \le (a + b + c + d)(\frac{1}{a} + \frac{1}{b} + \frac{1}{c} + \frac{1}{d})
    \end{gather*}
\end{proof}

\begin{homeworkProblem}{Axler 19}
    Prove $\langle u,v \rangle = \frac{\|u+v\|^2 - \|u-v\|^2}{4}$
\end{homeworkProblem}
\begin{proof}
    Take our proof for Axler 20 and set the complex component to $0$.
\end{proof}
\begin{homeworkProblem}{Axler 20}
    Prove $\langle u,v \rangle = \frac{\|u+v\|^2 - \|u-v\|^2 + \|u + iv\|^2i - \|u-iv\|^2i}{4}$
\end{homeworkProblem}
\begin{proof}
    I did this on paper and it took about a page, this is the summarized version.
    \begin{gather*}
    \langle u,v \rangle = \frac{\langle u+v,u+v \rangle - \langle u-v, u-v\rangle + \langle u+vi, u+vi \rangle i + \langle u-vi, u-vi \rangle i} {4} \\
    \langle u,v \rangle = \frac{4\langle u,v \rangle  + 2\langle u,vi \rangle i + 2\langle iv, u \rangle i} {4} \\
    \langle u,v \rangle = \frac{4\langle u,v \rangle} {4}
    \end{gather*}
\end{proof}
\end{document}

